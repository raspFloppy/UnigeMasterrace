\documentclass{article}


\begin{document}

\title{Logica}
\author{Floppy Loppy}
\date{September 2021}

\maketitle
\tableofcontents
\newpage


\section{Cos'e' la Logica}

La \textbf{logica} e' lo studio del ragionamento, e quando parliamo di \textbf{logica matematica} ci riferiamo allo studio del ragionamento matematico. \par
Il ragionamento ci è utile per la risoluzione dei problemi, attraverso la logica noi studiamo il ragionamento, e attraverso il ragionamento noi produciamo la logica. Si può dire che la \textbf{logica studia se stessa}.\par
In matematica ed informatica la logica la verità è stabilita da delle \textbf{dimostrazioni}. \par
Le dimostrazioni che risultano vere vengono definiti \textbf{teoremi}.

\subsection{Teorema}
Un teorema è composto da un'\textbf{ipotesi (o assunzione)} e da una \textbf{tesi}, l'ipotesi è  una o più assunzioni da cui partiamo mentre la tesi è la conseguenza del/delle ipotesi.


\subsection{Proposizione}

\section{Alberi}
Noi definiamo albero \par
Un albero è binario quando \par 
Un albero si dice etichettato quando ad ogni nodo è associato un elemento. \par

L'albero sintattico di una formula proposizionale P è l'albero binario etichettato finito tale che:

\begin{itemize}
        \item La radice è etichettata con $P$
        \item Se un nodo è etichettato con una formula  $Q$, allora:
                \begin{itemize}
                        \item se $Q$ è una formula atomica, allora tale nodo è una foglia
                        \item se $Q$ è (R), allora tale nodo ha un unico successore immediato, etichettato con R.
                        \item se $Q$ è ($R$ $S$).
                \end{itemize}
\end{itemize}


\subsection{Connettivo principale}


Il connettivo principale di un albero è qu

Per individuare il connettivo principale attraverso una formula \textbf{non atomica} $P$ dobbiamo verificare queste proprietà:

\begin{itemize}
        \item  

\end{itemize}



\end{document}
