\documentclass[../main.tex, class=article, 12pt]{subfiles}
\usepackage{float}
\usepackage{amsthm}
\usepackage{amsmath}
\usepackage{amssymb}
\usepackage{hyperref}
\usepackage{caption}
\usepackage{mathtools}
\usepackage{graphicx}
\usepackage{todonotes}
\usepackage{tcolorbox}
\graphicspath{{./images}}




\begin{document}
Possiamo considerare le tralsazioni come somme e le dilatazione-contrazioni come moltiplicazioni. 

\subsection{traslazioni verticali}\label{sec:traslazioni_verticali}
Consideriamo $ y = f(x) $ la sua traslazione verticale sarà la somma di un $ k $ con il valore di $ f(X) $ in quanto la funzione si sposterà sull'asse della $ y $. \par

Per ottenere $ \{y = f(x)+1\} $ devo traslare $   \{y = f(x)\} $ verso l'alto di $ 1 $ unità. \par
Per ottenere $ \{y = f(x)+a$  $\} $ devo traslare $   \{y = f(x)\} $ verso l'alto di $ |a| $ unità se $a > 0$ verso il basso di $ |a| $ unità se $ a < 0 $. \newline



\subsection{traslazioni orizzontali}\label{sec:traslazioni_orizzontali}
Consideriamo $ y = f(x) $ la sua traslazione orizzontale sarà la somma di un $ k $ con il valore di $ x $ in quanto la funzione si sposterà sull'asse della $ x $. \par

\todo{Aggiungere esempio traslazione orizzontale}


\subsection{dilatazioni e contrazioni verticali}\label{sec:dilatazioni_e_contrazioni_verticali}
Consideriamo $ y = f(x) $ la sua dilatazione e contrazione sarà il prodotto di un $ k $ con il valore di $ f(X) $ in quanto la funzione si dilaterà-contrarrà sull'asse della $ y $. \par


\todo{Aggiungere esempio dilatazioni e contrazioni verticali}


\subsection{dilatazioni e contrazioni orizzontali}\label{sec:dilatazioni_e_contrazioni_verticali}
\todo{Aggiungere esempio dilatazioni e contrazioni orizzontale}







\end{document}
