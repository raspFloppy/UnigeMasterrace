\documentclass[../main.tex, class=article, 12pt]{subfiles}
\graphicspath{{./images}}




\begin{document}
Sottoinsiemi di $ \mathbb{R} $ sono:
\begin{itemize}
        \item $ \mathbb{N} = \{0,1,2,3,\ldots\} $
        \item $ \mathbb{Z} = \{\ldots,-2,-1,0,1,2,\ldots\} $
        \item $ \mathbb{Q} = \{\frac{m}{n}: m \in \mathbb{Z} \wedge n \in \mathbb{N}^*\} $
\end{itemize}

\begin{tcolorbox}
       \begin{oss}
        In particolare $ \mathbb{Q} $ è detto \textbf{denso} ovvero  presi due qualunque punti $ x,y \in \mathbb{R} $ esiste sempre un razionale $ \mathbb{Q}  $ tra di essi.
       \end{oss} 
        \begin{exmp}
                Proviamo a dimostrarlo attraverso unaretta:
        \end{exmp}
\end{tcolorbox}
\todo{Aggiungere esempio con una retta}


\begin{tcolorbox}
\begin{prop}
Fondamentale Proprietà di $ \mathbb{R} $ è un insieme \textbf{totalmente ordinato}.
\end{prop}
\end{tcolorbox}

\begin{lemma}
       $ \forall x,y \in \mathbb{R} $ con $ x < y $ 
\end{lemma}


\subsection{Sottoinsiemi particolari di $\mathbb{R}$}\label{sec:sottoinsiemi_particolari_di_R}

Esistono diversi tipi di intervalli, elenchiamoli per categoria


\subsubsection{Intervalli limitati}
\begin{itemize}
        \item $ (a,b) = \{x \in \mathbb{R} : x>a \wedge x<b\} $ intervallo aperto
        \item $ [a,b] = \{x \in \mathbb{R} : x\ge a \wedge x\le b\} $ intervallo chuso
        \item $ [a,b) = \{x \in \mathbb{R} : x\ge a \wedge x<b\} $ intervallo semi-aperti
        \item $ (a,b] = \{x \in \mathbb{R} : x>a \wedge x\le b\} $ itervallo semi-aperti
\end{itemize}


\subsubsection{Intervalli illimitati}
Gli intervalli illimitati sono rappresentate geometricamente da semirette

\begin{itemize}
        \item $ (a,+\infty) = \{x \in \mathbb{R} : x>a \wedge x<+\infty\} $ intervallo aperto
        \item $ [a,+\infty] = \{x \in \mathbb{R} : x\ge a \wedge x\le +\infty\} $ intervallo chuso
        \item $ [-\infty,b) = \{x \in \mathbb{R} : x\ge -\infty \wedge x<b\} $ intervallo semi-aperti
        \item $ (-\infty,b] = \{x \in \mathbb{R} : x>-\infty \wedge x\le b\} $ itervallo semi-aperti
\end{itemize}


\subsection{Dominio e Codominio}\label{sec:}
\todo{recuperare lezione mannaggia il cazzo}

\begin{definition}
        (Funzione) Una funzione $ f:A \to R $ non è altro che una associazione \underline{univoca} di un elemento di $ A $ con uno di $ \mathbb{R} $. \par
        In particolare:
\begin{equation*}
        \forall x \in A \quad \exists! y\in\mathbb{R} : f(x) = g.
\end{equation*}
\end{definition}


\begin{tcolorbox}
\begin{prop}
       Una particolarità dei reali è che possiamo rappresentare il grafico della funzione:
\end{prop}
\end{tcolorbox}
\todo{Disegnare retta}
Come la retta rappresenta l'insieme $ \mathbb{R} $ il piano rappresenta l'insieme:
\begin{equation*}
        \mathbb{R}\times\mathbb{R} = \{(x,y): x\in \mathbb{R}, y \in \mathbb{R}\}
\end{equation*}
Il grafico di $ f $ non è altro che:
\begin{equation*}
        graph(f) = \{(x, f(x)) : x \in A = dom(f)\}\le \mathbb{R}^2
\end{equation*}

\begin{tcolorbox}
\begin{oss}
        Data una curva $ M \subseteq \mathbb{R}^2 $, essa è grafico di una funzione solo se $ \forall x \in \mathbb{R}$ esiste al più un punto $ y $ tale che $ (x,y) \in M $, cioè $ M $ intergetta le rette verticali al più di un punto
\end{oss}
\end{tcolorbox}
\todo{Inserire Esempio grafico 10:30}



\end{document}
