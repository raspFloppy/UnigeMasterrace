\documentclass[a4paper]{article}
\usepackage{tikz}
\usepackage{float}
\usepackage{amsthm}
\usepackage{amsmath}
\usepackage{amssymb}
\usepackage{caption}
\usepackage{hyperref}
\usepackage{graphicx}
\usepackage{pgfplots}
\usepackage{subfiles}
\usepackage{mathtools}
\usepackage{todonotes}
\usepackage{tcolorbox}
\graphicspath{{./images}}


\newtheorem{exmp}{Esempio}[section]
\newtheorem{theorem}{Teorema}[section]
\theoremstyle{definition}
\newtheorem{exer}{Esercizi}[section]
\newtheorem{definition}{Definizione}[section]
\newtheorem{oss}{Osservazione}[section]
\newtheorem{lemma}{Lemma}[section]
\newtheorem{corollario}{Corollario}[section]
\newtheorem{prop}{Proprietà}[section]
\newtheorem{nb}{Nota Bene}[section]


\begin{document}
\title{Appunti molto belli di Calculus}
\author{Floppy Loppy}
\date{March 2022}
\maketitle
\tableofcontents




\newpage
\section{Numeri Reali}\label{sec:introduzione}
\subfile{sections/reali.tex}


\newpage
\section{Proprietà Funzioni}\label{sec:proprietà_funzioni}
\subfile{sections/proprieta_funzioni.tex}

\newpage
\section{Trasformazioni geometriche}\label{sec:trasformazioni_gemoetriche}
\subfile{sections/trasformazioni_geometriche.tex}

\end{document}
