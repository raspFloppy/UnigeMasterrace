% !TeX root = main.tex
\documentclass[a4paper]{article}
\usepackage{tikz}
\usepackage{float}
\usepackage{amsthm}
\usepackage{amsmath}
\usepackage{amssymb}
\usepackage{caption}
\usepackage{paracol}
\usepackage{geometry}
\usepackage{multicol}
\usepackage{hyperref}
\usepackage{graphicx}
\usepackage{pgfplots}
\usepackage{subfiles}
\usepackage{mathtools}
\usepackage{todonotes}
\usepackage{tcolorbox}
\graphicspath{{./images}}
\setlength{\columnseprule}{0.2pt}

\newtheorem{exmp}{Esempio}[section]
\newtheorem{theorem}{Teorema}[section]
\theoremstyle{definition}
\newtheorem{exer}{Esercizi}[section]
\newtheorem{definition}{Definizione}[section]
\newtheorem{oss}{Osservazione}[section]
\newtheorem{lemma}{Lemma}[section]
\newtheorem{corollario}{Corollario}[section]
\newtheorem{prop}{Proprietà}[section]
\newtheorem{nb}{Nota Bene}[section]


\begin{document}
\title{Appunti molto belli di Calculus}
\author{Floppy Loppy}
\date{March 2022}
\maketitle
\tableofcontents
\newpage
\listoftodos/



\newpage
\section{Numeri Reali}\label{sec:introduzione}
\subfile{sections/reali.tex}


%
%
%\newpage
%\section{Proprietà Funzioni}\label{sec:proprietà_funzioni}
%\subfile{sections/proprieta_funzioni.tex}
%
%
%\newpage
%\section{Trasformazioni geometriche}\label{sec:trasformazioni_gemoetriche}
%\subfile{sections/trasformazioni_geometriche.tex}
%
%
%\newpage
%\section{Funzioni elementari}\label{sec:funzioni_elementari}
%\subfile{sections/funzioni_elementari.tex}
%
%
%\newpage
%\section{Limiti}\label{sec:limiti}
%\subfile{sections/limiti.tex}
%
%
%\newpage
%\section{Numero di Nepero}\label{sec:numero_di_nepero}
%\subfile{sections/nepero.tex}
%
%\newpage
%\section{Limiti inferiori e superiori}\label{sec:limiti_inferiori_superiori}
%\subfile{sections/inferiore_superiore.tex}
%
%\newpage
%\section{Derivata}\label{sec:derivata}
%\subfile{sections/derivata.tex}

\end{document}
