\documentclass[letterpaper 12pt]{article}
\usepackage{float}
\usepackage{amsthm}
\usepackage{amsmath}
\usepackage{amssymb}
\usepackage{hyperref}
\usepackage{caption}
\usepackage{mathtools}
\usepackage{graphicx}
\usepackage{tcolorbox}
\usepackage{subfiles}
\graphicspath{{./images}}

\newtheorem{exmp}{Esempio}[section]
\theoremstyle{definition}
\newtheorem{cit}{Citazione}[section]
\newtheorem{definition}{Definizione}[section]
\newtheorem{nb}{Nota Bene}[section]




\begin{document}

\title{Critical Thinking}
\author{Floppy Loppy}
\date{Feb, 2022}
\maketitle
\tableofcontents
\newpage




\section{Che cos'è il ragionamento}
\subfile{sections/section1.tex}


\newpage
\section{Tipi di ragionamento}
\subfile{sections/section2.tex}


\newpage
\section{Valutare un ragionamento}
\subfile{sections/section3.tex}


\newpage
\section{Fallacie}\label{sec:fallacie}
\subfile{sections/section4.tex}


\newpage
\section{Consigli di lettura}
\begin{itemize}
        \item Argomentazione Andrea Iacona
        \item Come non detto 
        \item verità avvelenata Agostini Bollati
        \item Critical Thinking Canale Ciuni Frigerio Tuzet
        \item pensieri lenti e veloci
\end{itemize}



\end{document}
