\documentclass[../main.tex, class=letterpaper]{subfiles}
\usepackage{float}
\usepackage{amsthm}
\usepackage{amsmath}
\usepackage{amssymb}
\usepackage{hyperref}
\usepackage{caption}
\usepackage{mathtools}
\usepackage{graphicx}
\usepackage{todonotes}
\usepackage{tcolorbox}
\graphicspath{{./images}}

\newtheorem{exmp}{Esempio}[section]
\theoremstyle{definition}
\newtheorem{cit}{Citazione}[section]
\newtheorem{definition}{Definizione}[section]
\newtheorem{nb}{Nota Bene}[section]


\begin{document}
Esistono tre tipi di ragionamenti:
\begin{itemize}
        \item Deduzione
        \item Induzione
        \item Abduzione
\end{itemize}

Questi tipi di ragionamento sono formati da:
\begin{itemize}
\item \textbf{REGOLA}.
\item \textbf{CASO}.
\item \textbf{RISULTATO}.
\end{itemize}



\subsection{Deduzione}
La deduzione è la forma di ragionamento più diffusa e conosciuta in quanto  tutti i ragionamenti deduttivi portano ad una conclusione certa.
\begin{definition}
        La deduzione e composta da:
        \begin{exmp}
                \begin{itemize}
                        \item\textbf{REGOLA}: tutti i fagioli di questo sacchetto sono bianchi.
                        \item\textbf{CASO}: Questi fagioli vengono da questo sacchetto.
                        \item \textbf{Risultato}: Questi fagioli sono bianchi. 
                \end{itemize}
        \end{exmp}
\end{definition}



\subsection{Induzione}
L'induzione porta ad un risultato non certo ma \textbf{probabile}, facciamo un esempio:
\begin{exmp}
        \begin{itemize}
                \item \textbf{RISULTATO}: Questi fagioli sono bianchi. 
                \item\textbf{CASO}: Questi fagioli vengono da questo sacchetto.
                \item\textbf{REGOLA}: tutti i fagioli (probabilmente) di questo sacchetto sono bianchi.
        \end{itemize}
\end{exmp}



\subsection{Abduzione}
Questo tipo di ragionamento è quello che solitamente mettiamo in pratica nella nostra vita quotidiana, vediamo un esempio:
\begin{exmp}
        \begin{itemize}
                \item \textbf{RISULTATO}: Questi fagioli sono bianchi. 
                \item\textbf{REGOLA}: tutti i fagioli di questo sacchetto sono bianchi.
                \item\textbf{CASO}: Questi fagioli vengono da questo sacchetto.
        \end{itemize}
\end{exmp}


\end{document}
