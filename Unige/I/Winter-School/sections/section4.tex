\documentclass[../main.tex, class=article, 12pt]{subfiles}
\usepackage{float}
\usepackage{amsthm}
\usepackage{amsmath}
\usepackage{amssymb}
\usepackage{hyperref}
\usepackage{caption}
\usepackage{mathtools}
\usepackage{graphicx}
\usepackage{todonotes}
\usepackage{tcolorbox}
\graphicspath{{./images}}

\newtheorem{exmp}{Esempio}[section]
\theoremstyle{definition}
\newtheorem{cit}{Citazione}[section]
\newtheorem{definition}{Definizione}[section]
\newtheorem{nb}{Nota Bene}[section]





\begin{document}
\begin{definition}
        Una fallacia è un ragionamento che sembra valido e corretto ma non lo è oppure essendo persuasivo e psicologicamente convincente non lo è. \par
\end{definition}

Esistono diversi tipi di fallacie:
\begin{itemize}
        \item \textbf{FALLACIE FORMALI}: quelle che violano le regole delle logiche di inferenza 
        \item \textbf{FALLACIE INFORMALI}: Sfruttano espedienti retorici o linguistici 
\end{itemize}



\subsection{Fallacie formali}\label{sec:fallacie_formali}
\begin{tcolorbox}
        Modus Ponens: \newline 
        Se $ P $ allora $ Q $ \newline
        $ P $ \newline
        $ Q $ \newline 
\end{tcolorbox} 
\begin{exmp}
        Esempio: \newline
        Se HO UNA CAROTA allora HO UNA VERDURA \newline
        HO UNA CAROTA \newline
        HO UNA VERDURA \newline 
\end{exmp}

\begin{tcolorbox}
        Affermazione del conseguente: \newline
        Se $ P $ allora $ Q $ \newline
        $ Q $ \newline
        $ P $ \newline 
\end{tcolorbox}
\begin{exmp}
        Esempio: \newline
        Se HO UNA CAROTA allora HO UNA VERDURA \newline
        HO UNA VERDURA \newline 
        HO UNA CAROTA \newline
\end{exmp}

\begin{tcolorbox}
        Modus Tollens\newline 
        Se $ P $ allora $ Q $ \newline
        non $ Q $ \newline 
        non $ P $ \newline
\end{tcolorbox}
\begin{exmp}
        Esempio: \newline
        Se HO UNA CAROTA allora HO UNA VERDURA \newline
        NON HO UNA VERDURA \newline 
        NON HO UNA CAROTA \newline
\end{exmp}

\begin{tcolorbox}
        Affermazione del conseguente: \newline
        Se $ P $ allora $ Q $ \newline
        non $ P $ \newline
        non $ Q $ \newline 
\end{tcolorbox}
\begin{exmp}
        Esempio: \newline
        Se HO UNA CAROTA allora HO UNA VERDURA \newline
        NON HO UNA CAROTA \newline
        NON HO UNA VERDURA \newline 
\end{exmp}



\subsection{Fallace Informali}
Sono errori di ragionamento che dipendono da una molteplicità di criteri. \par
\begin{itemize}
        \item \textbf{Rilevanza}
        \item \textbf{Semantica}
        \item\textbf{Induttive}
\end{itemize}
è il contenuto del ragionamento ad essere fallace in questo caso.



\subsubsection{Fallacia di rilevanza}
Quanto le premesse non hanno rilevanza per la conclusione ovvero in quei casi in cui la conclusione è proprio nella premessa:
\begin{exmp}
        Esempio: \newline
        Dio esiste perchè lo dice la Bibbia \newline
        Come fai a sapere che la Bibbia dice la verità? \newline
        Perchè è la parola di Dio \newline
\end{exmp}
Viene detto \textbf{ragionamento circolare}. \par

Possiamo trovare anche casi in cui la premessa non ha proprio conessioni con la conclusione:
\begin{exmp}
        Esempio: \newline
        Nessuno ha mai provato che gli OGM siano dannosi \newline
        Gli OGM non sono dannosi \newline
\end{exmp}
Oppure:
\begin{exmp}
        Esempio: \newline
        Tizio X famoso dice che i vaccini non funzionano\newline
        I vaccini non funzionano \newline
\end{exmp}



\subsubsection{Fallacia di semantica}
Le fallacie semantiche sono tutte quelle che vanno a peccare di valenza del significato. 



\subsubsection{Fallacia d'induzione}
Sono tutte quelle fallacie che vanno erroneamente ad utilizzare la probabilità dell'induzione
\begin{exmp}
 
        Quattro immigrati hanno aggredito una persona \newline
        Tutti gli immigrati sono criminali \newline
\end{exmp}



\end{document}
