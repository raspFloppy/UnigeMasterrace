\documentclass[../main.tex, class=article, 12pt]{subfiles}
\usepackage{float}
\usepackage{amsthm}
\usepackage{amsmath}
\usepackage{amssymb}
\usepackage{hyperref}
\usepackage{caption}
\usepackage{mathtools}
\usepackage{graphicx}
\usepackage{todonotes}
\usepackage{tcolorbox}
\graphicspath{{./images}}

\newtheorem{exmp}{Esempio}[section]
\theoremstyle{definition}
\newtheorem{cit}{Citazione}[section]
\newtheorem{definition}{Definizione}[section]
\newtheorem{nb}{Nota Bene}[section]


\begin{document}
Valutare un ragionamento può servire a vincere un dibattito correggento la propria argomentazione e trovare le fallacie~\ref{sec:fallacie} logiche dell'avversario. \par

\begin{tcolorbox}
Prima di tutto bisogna distiguere il problema della verità dalla validità.
\end{tcolorbox}

\subsection{Valida/Invalida}\label{sec:valida_invalida}
Un'argomentazione risulta valida se la conclusione segue le premesse. \newline
Un'argomentazione risulta invalida se la conclusione \textbf{NON} segue le premesse. \newline
\begin{exmp}
        Esempio di argomentazione valida: \newline
        \begin{itemize}
                \item Gli italiani sono mafiosi
                \item I milanesi sono italiani
                \item I milanesi sono mafiosi
        \end{itemize}
        Valida ma Scorretta in quanto le premesse sono false vedi~\ref{sec:corretta_scorretta}.
\end{exmp}
\begin{exmp}
        Esempio di argomentazione invalida: \newline
        \begin{itemize}
                \item Gli italiani sono mafiosi
                \item I milanesi sono mafiosi 
                \item I milanesi sono italiani 
        \end{itemize}
        La conclusione non segue la premessa in quanto non viene specificato che i milanesi sono italiani.
\end{exmp}



\subsection{Corretta/Scorretta}\label{sec:corretta_scorretta}
Un argomentazione valida si dice corretta se le premesse sono vere \newline
Un argomentazione valida si dice scorretta se le premesse non sono vere \newline
\begin{exmp}
        Esempio di un ragionamento corretto:
        \begin{itemize}
                \item O ti piace il pesto o giochi a calcio
                \item Non giochi a calcio
                \item Ti piace il pesto
        \end{itemize}
        Questa è corretta ma non risulta plausibile o convincente.
\end{exmp}
\begin{exmp}
        Esempio di un ragionamento scorretto è quello visto qui~\ref{sec:valida_invalida}
\end{exmp}



\subsection{Buona/Fallace}
Un buon ragionamento deve essere:
\begin{itemize}
        \item Valido.
        \item Corretto.
        \item Psicologicamente plausibile.
\end{itemize}
Se non lo è allora il ragionamento è \textbf{FALLACE} 
\begin{exmp}
        Un buon ragionamento è composto cosi':
        \begin{itemize}
                \item Gli italiani sono europei
                \item I milanesi sono italiani
                \item i milanesi sono europei
        \end{itemize}
\end{exmp}



\subsection{Respingere un ragionamento}
Un ragionamento che non suona plausibile è respingibile con un \textbf{controesempio} ovvero: \par
\begin{definition}
        Applicare lo stesso schema di inferenza usato nel ragionamento che sembra convincente con la stessa struttura di ragionamento.
\end{definition}
\begin{exmp}
        Costruiamo un controesempio: \newline
        \underline{Ragionamento errato}:
        \begin{itemize}
                \item I mafiosi sono siciliani
                \item i palermitato sono siciliani
                \item i palermitani sono mafiosi
        \end{itemize}

        \underline{Controesempio}:
        \begin{itemize}
                \item Gli italiani sono europei
                \item I francesi sono europei
                \item I francesi sono italiani 
        \end{itemize}
\end{exmp}


\end{document}
