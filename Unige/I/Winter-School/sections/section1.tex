\documentclass[../main.tex, class=letterpaper]{subfiles}
\usepackage{float}
\usepackage{amsthm}
\usepackage{amsmath}
\usepackage{amssymb}
\usepackage{hyperref}
\usepackage{caption}
\usepackage{mathtools}
\usepackage{graphicx}
\usepackage{tcolorbox}
\usepackage{subfiles}
\graphicspath{{./images}}

\newtheorem{exmp}{Esempio}[section]
\theoremstyle{definition}
\newtheorem{cit}{Citazione}[section]
\newtheorem{definition}{Definizione}[section]
\newtheorem{nb}{Nota Bene}[section]




\begin{document}

\begin{tcolorbox}
\begin{cit}
        \textbf{Aristotele:} Gli esseri umani sono animali razionali. 
\end{cit}
\end{tcolorbox}

\begin{definition}
       Ragionare significa \textbf{dare ragioni o svolgere inferenze}  
\end{definition}
\begin{definition}
        Inferenza: \newline
        \begin{itemize}
                \item L'azione di passare da delle premesse a una conclusione secondo regole, ragionando possiamo, partendo da nozioni gia note, arrivare a conclusioni secondo certe regole.
                \item La struttura di tale pasaggio è definito \textbf{Schema di inferenza}. 
        \end{itemize}
\end{definition}
Nella nostra vita quotidiana svolgiamo attività di inferenza in continuazione, seguendo anche regole di inferenza attivamente o passivamente.

\begin{exmp}
        Esempi di inferenze: \newline
        Se piove mi bagno \newline
        Piove \newline
        Mi bagno 
\end{exmp}

\begin{tcolorbox}
        Possiamo tradurlo logicamente con: \newline
        Se $ P $ allora $ Q $ \newline
        $ P $ \newline
        $ Q $ \newline 
\end{tcolorbox}
Questo scema di ragionamento si chiama \textbf{Modus Ponens} \par

L'opposto di questo schema si chiama \textbf{Modus Tollens} e si compone in questo modo:
\begin{tcolorbox}
        Possiamo tradurlo logicamente con: \newline
        Se $ P $ allora $ Q $ \newline
        non $ P $ \newline
        non $ Q $ \newline 
\end{tcolorbox}

\end{document}
