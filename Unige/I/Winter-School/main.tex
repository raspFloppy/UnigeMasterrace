\documentclass{article}
\usepackage{float}
\usepackage{amsthm}
\usepackage{amsmath}
\usepackage{amssymb}
\usepackage{hyperref}
\usepackage{caption}
\usepackage{mathtools}
\usepackage{graphicx}
\usepackage{todonotes}
\usepackage{tcolorbox}
\graphicspath{{./images}}

\newtheorem{exmp}{Esempio}[section]
\newtheorem{theorem}{Teorema}[section]
\theoremstyle{definition}
\newtheorem{cit}{Citazione}[section]
\newtheorem{definition}{Definizione}[section]
\newtheorem{oss}{Osservazione}[section]
\newtheorem{lemma}{Lemma}[section]
\newtheorem{corollario}{Corollario}[section]
\newtheorem{prop}{Proprietà}[section]
\newtheorem{nb}{Nota Bene}[section]




\begin{document}

\title{Crit}
\author{Floppy Loppy}
\date{September 2021}
\maketitle
\tableofcontents
\newpage



\section{Che cos'è il ragionamento}
\begin{tcolorbox}
        
\begin{cit}
        \textbf{Aristotele:} Gli esseri umani sono animali razionali. 
\end{cit}
\end{tcolorbox}
\begin{definition}
       Ragionare significa \textbf{dare ragioni o svolgere inferenze}  
\end{definition}
\begin{definition}
        Inferenza: \newline
        \begin{itemize}
                \item L'azione di passare da delle premesse a una conclusione secondo regole, ragionando possiamo, partendo da nozioni gia note, arrivare a conclusioni secondo certe regole.
                \item La struttura di tale pasaggio è definito \textbf{Schema di inferenza}. 
        \end{itemize}
\end{definition}
Nella nostra vita quotidiana svolgiamo attività di inferenza in continuazione, seguendo anche regole di inferenza attivamente o passivamente.

\begin{exmp}
        Esempi di inferenze: \newline
        Se piove mi bagno \newline
        Piove \newline
        Mi bagno 
\end{exmp}

\begin{tcolorbox}
        Possiamo tradurlo logicamente con: \newline
        Se $ P $ allora $ Q $ \newline
        $ P $ \newline
        $ Q $ \newline 
\end{tcolorbox}
Questo scema di ragionamento si chiama \textbf{Modus Ponens} \par

L'opposto di questo schema si chiama \textbf{Modus Tollens} e si compone in questo modo:
\begin{tcolorbox}
        Possiamo tradurlo logicamente con: \newline
        Se $ P $ allora $ Q $ \newline
        non $ P $ \newline
        non $ Q $ \newline 
\end{tcolorbox}






\newpage
\section{Tipi di ragionamento}
Esistono tre tipi di ragionamenti:
\begin{itemize}
        \item Deduzione
        \item Induzione
        \item Abduzione
\end{itemize}

Questi tipi di ragionamento sono formati da:
\begin{itemize}
\item \textbf{REGOLA}.
\item \textbf{CASO}.
\item \textbf{RISULTATO}.
\end{itemize}



\subsection{Deduzione}
La deduzione è la forma di ragionamento più diffusa e conosciuta in quanto  tutti i ragionamenti deduttivi portano ad una conclusione certa.
\begin{definition}
        La deduzione e composta da:
        \begin{exmp}
                \begin{itemize}
                        \item\textbf{REGOLA}: tutti i fagioli di questo sacchetto sono bianchi.
                        \item\textbf{CASO}: Questi fagioli vengono da questo sacchetto.
                        \item \textbf{Risultato}: Questi fagioli sono bianchi. 
                \end{itemize}
        \end{exmp}
\end{definition}



\subsection{Induzione}
L'induzione porta ad un risultato non certo ma \textbf{probabile}, facciamo un esempio:
\begin{exmp}
        \begin{itemize}
                \item \textbf{RISULTATO}: Questi fagioli sono bianchi. 
                \item\textbf{CASO}: Questi fagioli vengono da questo sacchetto.
                \item\textbf{REGOLA}: tutti i fagioli (probabilmente) di questo sacchetto sono bianchi.
        \end{itemize}
\end{exmp}



\subsection{Abduzione}
Questo tipo di ragionamento è quello che solitamente mettiamo in pratica nella nostra vita quotidiana, vediamo un esempio:
\begin{exmp}
        \begin{itemize}
                \item \textbf{RISULTATO}: Questi fagioli sono bianchi. 
                \item\textbf{REGOLA}: tutti i fagioli di questo sacchetto sono bianchi.
                \item\textbf{CASO}: Questi fagioli vengono da questo sacchetto.
        \end{itemize}
\end{exmp}






\newpage
\section{Valutare un ragionamento}
Valutare un ragionamento può servire a vincere un dibattito correggento la propria argomentazione e trovare le fallacie~\ref{sec:fallacie} logiche dell'avversario. \par

\begin{tcolorbox}
Prima di tutto bisogna distiguere il problema della verità dalla validità.
\end{tcolorbox}

\subsection{Valida/Invalida}\label{sec:valida_invalida}
Un'argomentazione risulta valida se la conclusione segue le premesse. \newline
Un'argomentazione risulta invalida se la conclusione \textbf{NON} segue le premesse. \newline
\begin{exmp}
        Esempio di argomentazione valida: \newline
        \begin{itemize}
                \item Gli italiani sono mafiosi
                \item I milanesi sono italiani
                \item I milanesi sono mafiosi
        \end{itemize}
        Valida ma Scorretta in quanto le premesse sono false vedi~\ref{sec:corretta_scorretta}.
\end{exmp}
\begin{exmp}
        Esempio di argomentazione invalida: \newline
        \begin{itemize}
                \item Gli italiani sono mafiosi
                \item I milanesi sono mafiosi 
                \item I milanesi sono italiani 
        \end{itemize}
        La conclusione non segue la premessa in quanto non viene specificato che i milanesi sono italiani.
\end{exmp}



\subsection{Corretta/Scorretta}\label{sec:corretta_scorretta}
Un argomentazione valida si dice corretta se le premesse sono vere \newline
Un argomentazione valida si dice scorretta se le premesse non sono vere \newline
\begin{exmp}
        Esempio di un ragionamento corretto:
        \begin{itemize}
                \item O ti piace il pesto o giochi a calcio
                \item Non giochi a calcio
                \item Ti piace il pesto
        \end{itemize}
        Questa è corretta ma non risulta plausibile o convincente.
\end{exmp}
\begin{exmp}
        Esempio di un ragionamento scorretto è quello visto qui~\ref{sec:valida_invalida}
\end{exmp}



\subsection{Buona/Fallace}
Un buon ragionamento deve essere:
\begin{itemize}
        \item Valido.
        \item Corretto.
        \item Psicologicamente plausibile.
\end{itemize}
Se non lo è allora il ragionamento è \textbf{FALLACE} 
\begin{exmp}
        Un buon ragionamento è composto cosi':
        \begin{itemize}
                \item Gli italiani sono europei
                \item I milanesi sono italiani
                \item i milanesi sono europei
        \end{itemize}
\end{exmp}



\subsection{Respingere un ragionamento}
Un ragionamento che non suona plausibile è respingibile con un \textbf{controesempio} ovvero: \par
\begin{definition}
        Applicare lo stesso schema di inferenza usato nel ragionamento che sembra convincente con la stessa struttura di ragionamento.
\end{definition}
\begin{exmp}
        Costruiamo un controesempio: \newline
        \underline{Ragionamento errato}:
        \begin{itemize}
                \item I mafiosi sono siciliani
                \item i palermitato sono siciliani
                \item i palermitani sono mafiosi
        \end{itemize}

        \underline{Controesempio}:
        \begin{itemize}
                \item Gli italiani sono europei
                \item I francesi sono europei
                \item I francesi sono italiani 
        \end{itemize}
\end{exmp}






\newpage
\section{Fallacie}\label{sec:fallacie}
\begin{definition}
        Una fallacia è un ragionamento che sembra valido e corretto ma non lo è oppure essendo persuasivo e psicologicamente convincente non lo è. \par
\end{definition}

Esistono diversi tipi di fallacie:
\begin{itemize}
        \item \textbf{FALLACIE FORMALI}: quelle che violano le regole delle logiche di inferenza 
        \item \textbf{FALLACIE INFORMALI}: Sfruttano espedienti retorici o linguistici 
\end{itemize}

\subsection{Fallacie formali}
\begin{tcolorbox}
        Modus Ponens: \newline 
        Se $ P $ allora $ Q $ \newline
        $ P $ \newline
        $ Q $ \newline 
\end{tcolorbox} 
\begin{exmp}
        Esempio: \newline
        Se HO UNA CAROTA allora HO UNA VERDURA \newline
        HO UNA CAROTA \newline
        HO UNA VERDURA \newline 
\end{exmp}

\begin{tcolorbox}
        Affermazione del conseguente: \newline
        Se $ P $ allora $ Q $ \newline
        $ Q $ \newline
        $ P $ \newline 
\end{tcolorbox}
\begin{exmp}
        Esempio: \newline
        Se HO UNA CAROTA allora HO UNA VERDURA \newline
        HO UNA VERDURA \newline 
        HO UNA CAROTA \newline
\end{exmp}

\begin{tcolorbox}
        Modus Tollens\newline 
        Se $ P $ allora $ Q $ \newline
        non $ Q $ \newline 
        non $ P $ \newline
\end{tcolorbox}
\begin{exmp}
        Esempio: \newline
        Se HO UNA CAROTA allora HO UNA VERDURA \newline
        NON HO UNA VERDURA \newline 
        NON HO UNA CAROTA \newline
\end{exmp}


\begin{tcolorbox}
        Affermazione del conseguente: \newline
        Se $ P $ allora $ Q $ \newline
        non $ P $ \newline
        non $ Q $ \newline 
\end{tcolorbox}
\begin{exmp}
        Esempio: \newline
        Se HO UNA CAROTA allora HO UNA VERDURA \newline
        NON HO UNA CAROTA \newline
        NON HO UNA VERDURA \newline 
\end{exmp}



\subsection{Fallace Informali}
Sono errori di ragionamento che dipendono da una molteplicità di criteri. \par
\begin{itemize}
        \item \textbf{Rilevanza}
        \item \textbf{Semantica}
        \item\textbf{Induttive}
\end{itemize}
è il contenuto del ragionamento ad essere fallace in questo caso.


\subsubsection{Fallacia di rilevanza}
Quanto le premesse non hanno rilevanza per la conclusione ovvero in quei casi in cui la conclusione è proprio nella premessa:
\begin{exmp}
        Esempio: \newline
        Dio esiste perchè lo dice la Bibbia \newline
        Come fai a sapere che la Bibbia dice la verità? \newline
        Perchè è la parola di Dio \newline
\end{exmp}
Viene detto \textbf{ragionamento circolare}. \par

Possiamo trovare anche casi in cui la premessa non ha proprio conessioni con la conclusione:
\begin{exmp}
        Esempio: \newline
        Nessuno ha mai provato che gli OGM siano dannosi \newline
        Gli OGM non sono dannosi \newline
\end{exmp}
Oppure:
\begin{exmp}
        Esempio: \newline
        Tizio X famoso dice che i vaccini non funzionano\newline
        I vaccini non funzionano \newline
\end{exmp}



\subsection{Fallacia di semantica}
Le fallacie semantiche sono tutte quelle che vanno a peccare di valenza del significato. 


\subsection{Fallacia d'induzione}
Sono tutte quelle fallacie che vanno erroneamente ad utilizzare la probabilità dell'induzione
\begin{exmp}
        Quattro immigrati hanno aggredito una persona \newline
        Tutti gli immigrati sono criminali \newline
\end{exmp}


\newline
\section{Consigli di lettura}
\begin{itemize}
        \item Argomentazione Andrea Iacona
        \item Come non detto 
        \item verità avvelenata Agostini Bollati
\end{itemize}
\end{document}
