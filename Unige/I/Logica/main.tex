\documentclass{article}
\usepackage{amsmath}
\usepackage{amssymb}
\usepackage{mathtools}
\usepackage{hyperref}
\usepackage{caption}
\usepackage{todonotes}
\usepackage{graphicx}

\graphicspath{{.images}}


\begin{document}

\title{Logica}
\author{Floppy Loppy}
\date{September 2021}

\maketitle
\tableofcontents
\listoftodos\



\newpage
\section{Cos'e' la Logica}
La \textbf{logica} e' lo studio del ragionamento, e quando parliamo di \textbf{logica matematica} ci riferiamo allo studio del ragionamento matematico. \par
Il ragionamento ci è utile per la risoluzione dei problemi, attraverso la logica noi studiamo il ragionamento, e attraverso il ragionamento noi produciamo la logica. Si può dire che la \textbf{logica studia se stessa}.\par
In matematica ed informatica la logica la verità è stabilita da delle \textbf{dimostrazioni}. \par
Le dimostrazioni che risultano vere vengono definiti \textbf{teoremi}.


\subsection{Teorema}
Un teorema è composto da un'\textbf{ipotesi (o assunzione)} e da una \textbf{tesi}, l'ipotesi è  una o più assunzioni da cui partiamo mentre la tesi è la conseguenza del/delle ipotesi.


\subsection{Proposizione}
\todo{Aggiungere DEF proposizione}




\newpage
\section{Logica proposizionale}
La logica proposizionale viene definita attraverso delle \textbf{interpretazioni} ed il significato è ottenibile attraverso una tavola di verità. \todo{Completare DEF logica proposizionale}\par




\newpage
\section{Alberi}
Noi definiamo albero \todo{Aggiungere DEF di albero }\par
Un albero è binario quando \todo{Aggiungere DEF di albero binario}\par 
Un albero si dice etichettato quando ad ogni nodo è associato un elemento. \par
L'albero sintattico di una formula proposizionale P è l'albero binario etichettato finito tale che:
\begin{itemize}
        \item La radice è etichettata con $P$
        \item Se un nodo è etichettato con una formula  $Q$, allora:
                \begin{itemize}
                        \item se $Q$ è una formula atomica, allora tale nodo è una foglia
                        \item se $Q$ è (R), allora tale nodo ha un unico successore immediato, etichettato con R.
                        \item se $Q$ è ($R$ $S$).
                \end{itemize}
\end{itemize}


\subsection{Connettivo principale}
Il connettivo principale di un albero è \todo{Aggiungere DEF di connettivo di albero} 
Per individuare il connettivo principale attraverso una formula \textbf{non atomica} $P$ dobbiamo verificare queste proprietà:




\newpage
\section{Logica del prim'ordine}
Un linguaggio di prim'ordine è composto da:
\begin{itemize}
        \item Una certa quantità di \textbf{simboli di relazione} ($ <,\le,\ldots $).
        \item Una certa quantità di \textbf{simboli di funzionali}. ($ +,*,\ldots $).
        \item Una certa quantità di \textbf{simboli di costante}. ($ 0,1,2,\ldots,\pi,\ldots $).
\end{itemize}
Con la logica del prim'ordine noi andiamo 


\subsection{Contesto nella logica del prim'ordine}
Una $ L$-struttura  $ A $ consiste di:
\begin{itemize}
        \item Un insieme non vuoto $ |A| $ detto \textit{universo} o \textit{dominio} della struttura.
        \todo{Aggiungere esempi}
\end{itemize}

 \begin{equation*}
        \mathcal{A} = (A, P^{\mathcal{A}}) \quad  \mbox{ha infiniti contesti interpretabili dove} \quad A \not = \emptyset \quad P^{\mathcal{A}} \subseteq A*A 
\end{equation*}\todo{Spiegare la formula}





\end{document}
