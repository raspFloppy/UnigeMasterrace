\documentclass{article}


\begin{document}

\title{Logica}
\author{Floppy Loppy}
\date{September 2021}
\maketitle



\begin{abstract}

\section{Cos'e' la Logica}

La \textbf{logica} e' lo studio del ragionamento, e quando parliamo di \textbf{logica matematica} ci riferiamo allo studio del ragionamento matematico. \par
Il ragionamento ci è utile per la risoluzione dei problemi, attraverso la logica noi studiamo il ragionamento, e attraverso il ragionamento noi produciamo la logica. Si può dire che la \textbf{logica studia se stessa}.\par
In matematica ed informatica la logica la verità è stabilita da delle \textbf{dimostrazioni}. \par
Le dimostrazioni che risultano vere vengono definiti \textbf{teoremi}.

\subsection{Teorema}
Un teorema è composto da un'\textbf{ipotesi (o assunzione)} e da una \textbf{tesi}, l'ipotesi è  una o più assunzioni da cui partiamo mentre la tesi è la conseguenza del/delle ipotesi.


\subsection{Proposizione}


\end{abstract}



\end{document}
