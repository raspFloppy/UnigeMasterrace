\documentclass{article}
\usepackage{amsmath}
\usepackage{hyperref}
\usepackage{caption}



\begin{document}

\title{Appunti belli di Algebra}
\author{Floppy Loppy}
\date{September 2021}
\maketitle

\tableofcontents
\newpage


\section{Insiemi}
Noi definiamo \textbf{insieme} una \textbf{collezione} di elementi, questi elementi possono qualsiasi cosa: numeri, oggetti, persone, ecc.. \par
Gli elementi fanno parte di un insieme soltanto se rispettano le proprietà dell'insieme stesso, per esempio gli elementi dell'insieme dei numeri pari dovranno avere come proprietà quella di essere pari appunto. \par
Perfetto ora che abbiamo una definizione di insieme possiamo iniziare ad introdurre la sintassi e alcune proprietà.

\subsection{Proprietà degli insiemi}
Consideriamo di avere un insieme di nome \textit{A} e un elemento che chiamiamo \textit{x} che fa parte di \textit{A} (perchè rispetta le proprietà dell'insieme), allora si dice che \textit{x} \textbf{Appartiene} ad \textit{A}, ciò in Algebra si scrive:
\begin{equation}
        x \in A
\end{equation}

Mentre l'opposto ovvero che un elemento \textit{x} non fa parte di \textit{A} (perchè non rispetta le proprietà dell'insieme), allora si dice \textit{x} \textbf{Non Appartiene} ad \textit{A}, e ciò in si scrive (Nella lingua degli algebristi):
\begin{equation}
        x \not \in A
\end{equation}

Se un insieme ha più di un elemento, che possono essere \{\textit{x1,x2,\ldots,$x_n$}\} allora possiamo sintetizzare la scrittura del fatto che ognuno di questi elementi appartiene all'insieme \textit{A} scrivendo:
\begin{equation}
        x = \{x1,x2,\ldots,x_n\} 
\end{equation}

Oppure (visto che piace ai matematici) sintetizzare ancora di più scrivendo:
\begin{equation}
        A = \{x : P(x)\}
\end{equation}

Che si legge \textit{A uguale agli elementi di x tali che $P(x)$}, dove:
\begin{itemize}
        \item x sono gli elementi.
        \item $P(x)$ la proprietà dell'insieme \textit{A} che gli elementi di \textit{A} devono rispettare.
\end{itemize}
La proprietà $P(x)$ ha l'obbligo di essere \textbf{oggettiva} ovvero in grado di dare un valore oggettivamente vero o falso ad un elemento. \par
Possiamo utilizzare un esempio più concreto come può essere quello dei numeri pari scriviendo:
\begin{equation}
        A = \{x : x \quad \textrm{è un numero pari} \} 
\end{equation}

In questo caso possiamo dire che:
\begin{align*}
        2 \in A \\
        3 \not \in A \\
        Alessio \not \in A \\ 
\end{align*}
In quanto 2 è pari perciò appartiene ad A, 3 è dispari quindi non appartiene all'insieme e Alessio non è un numero pari quindi non può appartenere all'insieme descritto.

Questo perchè la proprietà di essere pari è \textbf{oggettiva} mentre per esempio:
\begin{equation}
        B = \{x : x \quad \textrm{è un libro interessante} \} 
\end{equation}
Non può essere un insieme in quanto essere un \textit{libro interessante} non è una proprietà oggettiva.

Proseguendo possiamo trovare anche insiemi che contengono un solo elemento, questi insiemi sono detti \textbf{singoletti} e sono scritti:
\begin{equation}
        \{*\}
\end{equation}
Dove * rappresenta il singolo elemento.

Ed infine, l'insieme vuoto che si rappresente con il simbolo:
\begin{equation}
        \emptyset 
\end{equation}
Spiegandolo brevemente questo insieme non contiene nessun elemento (infatti si definisce vuoto), e possiede alcune proprietà interessanti come per esempio quello di essere contenuto in qualsiasi insieme.


\newpage


\subsection{Connettivi Logici}
Attraverso quelli che chiamiamo \textbf{connettivi logici} possiamo eseguire delle operazioni tra insiemi, da queste operazioni noi possiamo ricavare due valori: vero o falso,  andiamone a vederne alcune. \par

Prima di tutto definiamo due \textbf{proposizioni/affermazioni} fittizzie che chiamiamo $P$ e $D$ e partendo da questi andiamo a scrivere le operazioi che si possono effettuare su di essi:
\begin{itemize}
        \item La \textbf{Disgiunzione} scritta: $P \vee D$ ha valore vero quando almeno una delle due proposizione risulta vera, se entrambe sono false avremo invece un valore falso. 
        \item La \textbf{Congiunzione} scritta: $P \wedge D$ ha valore vero solo quando entrambe sono vere altrimenti otteniamo un valore falso.
        \item La \textbf{Negazione} scritta: $\lnot P$ inverte il valore della propsizione, se infatti P è vera $\lnot P$ sarà falsa e viceversa.
        \item L' \textbf{Implicazione} scritta: $P \Rightarrow D$ ha valore vero solo quando D è vera.
        \item L' \textbf{Equivalenza} scritta: $P \Leftrightarrow D$ ha valore vero solo quando P e D hanno lo stesso valore logico (vero;vero), (falso;falso).
\end{itemize}


\subsection{Quantificatori universali}
Abbiamo poi quelli che si chiamano quantificatori universali che servono a descrivere le proposizioni e le andremo a spiegare partendo da una proposizione qualsiasi che chiameremo $P$. \\

Scriviamo:
\begin{equation}
P : \forall x \in A  
\end{equation}
per dire che \textbf{per ogni} elemento di $A$ la proposizione $P$ vale. \\

Mentre scriviamo:
\begin{equation}
P : \exists x \in A  
\end{equation}
Per dire che \textbf{esiste almeno} un elemento di $A$ tale per cui la proposizione $P$ è vera. \\

Possiamo fare un esempio concreto, prendiamo un insieme $A = \{2,4,6,8\}$ e $P(x) = \textrm{x + 2 è pari}$ da questo possiamo dire con certezza che: 
\begin{align}
        \forall x \in A \quad P(x) \quad \textrm{è vera in quanto ogni elemento di A è pari} \\
        \exists x \in A \quad P(x) \quad \textrm{è vera in quanto almeno un elemento di A è pari}
\end{align}
\\

Abbiamo poi \textbf{l'esiste unico} che sta ad indicare che esiste un solo elemento in un dato insieme affinché una proposizione risulti vera:
\begin{equation}
        \exists! x \in A
\end{equation}


\subsection{Definizioni}
Ora andiamo ad introdurre alcune definizioni della teoria degli insiemi prendendo due insiemi fittizzi $A$ e $B$. \\

Si dice che $A$ è contenuto in $B$ se:
\begin{equation}
        \{\forall x \in A: x \in B\}
\end{equation}
e si legge \textit{per tutti gli elementi di A sono elementi di B} e lo scriviamo in questo modo:
\begin{equation}        
        A \subseteq B        
\end{equation}
ovvero \textit{A sottoinsieme di B oppure A contenuto in B}. \newline

Poi abbiamo $A$ uguale a $B$ se:
\begin{equation}
        x \in A \Leftrightarrow x \in B
\end{equation}
ovvero \textit{ogni elemento x appartiene sia ad A che a B}. \newline

Troviamo poi \textbf{l'unione} tra due insiemi: 
\begin{equation}
        A \cup B
\end{equation}
che sta a significare che ogni elemento di A appartene anche a B, scritto in matematichese:
\begin{equation}
        A \cup B = \{x : (x \in A) \vee (x \in B)\}
\end{equation} \newline


Mentre \textbf{l'intersezione} che rappresenta l'insieme degli elementi in comune tra due insiemi si scrive: 
\begin{equation}
        A \cap B
\end{equation}
e significa:
\begin{equation}
        A \cap B = \{x : (x \in A) \wedge (x \in B)\}
\end{equation} \newline

Infine abbiamo la \textbf{differenza o complementare} che è praticamente una sottrazione tra insiemi si scrive:
\begin{equation}
        B \setminus A = {x : (x \in B) \wedge (x \notin A)}
\end{equation}
ovvero tutti gli elementi di $B$ che non appartengono ad $A$, spiegato meglio si tolgono a $B$ gli elementi che fanno parte di $A$. \newline

Ma noi vogliamo esempi pratici giusto?, ok e  allora prendiamo due insiemi: $A = \{1,2,4\}$ e $B = \{1,2,3,4,5\}$ avremo che:
\begin{align}
        A \subseteq B \quad \textrm{vero} \\
        A = B \quad \textrm{falso} \\
        A \cup B = \{1, 2, 3, 4, 5\} \quad \textrm{oppure} \quad A \cup B = B \\
        A \cap B = \{1, 2, 4\} \quad \textrm{oppure} \quad A \cap B = A \\ 
        B \setminus A = \{3, 4, 5\}
\end{align}


\subsection{Insieme delle parti}
L'insieme delle parti è l'insieme dei sottoinsiemi contenuti in un dato insieme, ok spieghiamolo meglio, l'insieme delle parti di un insieme $A$ è l'insieme degli elementi che sono sottoinsiemi dell'insieme $A$. \par
Se la cosa vi confonde ancora facciamo un esempio concreto, prendiamo un insieme $A = \{1,2,3\}$ l'insieme delle parti, che si scrive $P(A)$ è:

\begin{equation}
        P(A) = \{\emptyset, \{1\}, \{2\}, \{3\}, \{1, 2\}, \{1, 3\}, \{2, 3\}, A \}
\end{equation}
Adesso il concetto dovrebbe essere (spero), più chiaro. \newline 

Prendiamo un esempio particolare dell'insieme delle parti, l'insieme delle parti dell'insieme vuoto, come sappiamo infatti l'insieme vuoto non ha nessun elemento, ma l'insieme delle parti è differente è l'insieme dei sotto insiemi di un dato insieme e come sappiamo ogni insieme ha come elemento l'insieme vuoto perciò:
\begin{equation}
        P\{\emptyset\} = \{\emptyset  \}
\end{equation}



\end{document}
