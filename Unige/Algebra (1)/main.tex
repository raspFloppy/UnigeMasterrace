\documentclass{article}



\begin{document}

\title{Appunti belli di Algebra}
\author{Floppy Loppy}
\date{September 2021}
\maketitle

\tableofcontents
\newpage


\section{Insiemi}
Noi definiamo \textbf{insieme} una \textbf{collezione} di elementi, questi elementi possono qualsiasi cosa: numeri, oggetti, persone, ecc.. \par
Gli elementi fanno parte di un insieme soltanto se rispettano le proprietà dell'insieme stesso, per esempio gli elementi dell'insieme dei numeri pari dovranno avere come proprietà quella di essere pari appunto. \par
Perfetto ora che abbiamo una definizione di insieme possiamo iniziare ad introdurre la sintassi e alcune proprietà.

\subsection{Proprietà degli insiemi}
Consideriamo di avere un insieme di nome \textit{A} e un elemento che chiamiamo \textit{x} che fa parte di \textit{A} (perchè rispetta le proprietà dell'insieme), allora si dice che \textit{x} \textbf{Appartiene} ad \textit{A}, ciò in Algebra si scrive:
\begin{equation}
        x \in A
\end{equation}

Mentre l'opposto ovvero che un elemento \textit{x} non fa parte di \textit{A} (perchè non rispetta le proprietà dell'insieme), allora si dice \textit{x} \textbf{Non Appartiene} ad \textit{A}, e ciò in si scrive (Nella lingua degli algebristi):

\begin{equation}
        x \not \in A
\end{equation}

Se un insieme ha più di un elemento, che possono essere \{\textit{x1,x2,\ldots,$x_n$}\} allora possiamo sintetizzare la scrittura del fatto che ognuno di questi elementi appartiene all'insieme \textit{A} scrivendo:
\begin{equation}
        x = \{x1,x2,\ldots,x_n\} 
\end{equation}

Oppure (visto che piace ai matematici) sintetizzare ancora di più scrivendo:
\begin{equation}
        A = \{x : P(x)\}
\end{equation}

Che si legge \textit{A uguale agli elementi di x tali che $P(x)$}, dove:
\begin{itemize}
        \item x sono gli elementi.
        \item $P(x)$ la proprietà dell'insieme \textit(A) che gli elementi di \textit{A} devono rispettare.
\end{itemize}




\end{document}
